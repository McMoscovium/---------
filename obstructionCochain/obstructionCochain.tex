\documentclass[a4paper,11pt]{jsarticle}


\usepackage{amsmath,amsthm,amsfonts,float,cases,bm,amssymb,amssymb,ascmac,url,color,enumerate}
\usepackage[dvipdfmx]{graphicx}
\usepackage[all]{xy}
\usepackage{makeidx}
\usepackage[bbgreekl]{mathbbol}
\usepackage[%
 dvipdfmx,% 欧文ではコメントアウトする
 setpagesize=false,%
 bookmarks=true,%
 bookmarksdepth=tocdepth,%
 bookmarksnumbered=true,%
 colorlinks=false,%
 pdftitle={},%
 pdfsubject={},%
 pdfauthor={},%
 pdfkeywords={}%
]{hyperref}
% PDFのしおり機能の日本語文字化けを防ぐ((u)pLaTeXのときのみかく)
\usepackage{pxjahyper}
\DeclareMathSymbol{\bbepsilon}{\mathord}{bbold}{"0F}



\setlength{\textwidth}{\fullwidth}
\setlength{\textheight}{40\baselineskip}
\addtolength{\textheight}{\topskip}
\setlength{\voffset}{-0.55in}

\theoremstyle{definition}
\newtheorem{thm}{定理}[section]
\newtheorem{prop}[thm]{命題}
\newtheorem{cor}[thm]{系}
\newtheorem{dfn}[thm]{定義}
\newtheorem{lem}[thm]{補題}
\newtheorem{rem}[thm]{注意}
\newtheorem{eg}[thm]{例}

\DeclareMathOperator{\Hom}{\mathrm{Hom}}
\DeclareMathOperator{\id}{\mathrm{id}}
\DeclareMathOperator{\Int}{\mathrm{Int}}

\setcounter{tocdepth}{3}%目次に表示する数字の深さ
\setcounter{section}{-1}

\begin{document}
\textgt{編集メモ}
\begin{itemize}
  \item このメモは昼間眠くないときにこのpdfを読んで気づいたことをまとめたものである.夜に編集すると勢いでやっつけ仕事のようになる場合がある.
  \item 現状分かりにくい.\begin{itemize}
    \item 「便利な写像」は便利だが,基本群のホモトピー群への作用をここまで具体的に書かなくても後の議論は回るのでは?
    \item 絶対ホモトピー群は$[(S^n,*),(X,*)]_0$を採用していながら相対ホモトピー群は$[(I^n,\partial I^n,J^{n-1}),(X,A,*)]_0$を採用している部分がある.後者のcubeを使う方法にした方が対のホモトピー完全列を記述しやすい.
  \end{itemize}
\end{itemize}

\date{}
\title{Obstruction Cochain}

\maketitle

\begin{abstract}
  Obstruction cochainが特性写像のとりかたに依らないこと,およびそれがcocycleであることの証明を述べた,しかし内容は{\HUGE\bfseries \color{red} 保証できない}ノートである.
\end{abstract}
\tableofcontents
\section{基本群のホモトピー群への作用}
\subsection{Transport}
この節と次の節はtammo tom Dieckを参考することにした.

\paragraph{基本群へのホモトピー群への作用}
$u\colon I\to X$, $f\colon(I^n,\partial I^n)\to (X,\gamma(0))$が与えられたとする.このとき,$u$は2つの定置写像$f|_{\partial I^n}=c_{\gamma(0)},\ c_{\gamma(1)}\colon \partial I^n\to X$の間のホモトピー$\hat u\colon \partial I^n\times I\to X$を引き起こす.$(I^n,\partial I^n)$はNDR-pairなのでホモトピー$\hat u$は初期条件を$f$として$I^n$上のホモトピー$U\colon I^n\times I\to X$に延ばせる.

$U_1$は$f$とホモトピックで,とくにspheroid $U_1\colon (I^n,\partial I^n)\to (X,u(1))$を定める.$U$による境界$\partial I^n$の像は$u$に沿っているので,この$U_1$は$f$を$u$に沿って「ずらした」ものと思える.このpdfでのみ,$y_0,\ y_1\in Y$をつなぐ道$\alpha$と2つの写像$f\colon (X,x_0)\to (Y,y_0),\ g\colon (X,x_0)\to (Y,y_1)$に対し,$f,g$の間のホモトピー$h\colon X\times I\to Y$が$h(x_0,t)=\alpha(t)$を満たすとき,$h$を,$\alpha$に沿っていると呼ぶことにする.

$u$としてループをとると$f$と$U_1$のcodomainは同じ基点を持つ.これによりホモトピー群への基本群の作用が定まることがこの後の議論によりわかる.まずは絶対ホモトピー群に絞って考える.
\begin{prop}(絶対ホモトピー群に値を持つtransport functor)
  \begin{enumerate}[(1)]
    \item 対応$[f]=[U_0]\mapsto[U_1]$はwell-definedな準同型
    \[
      u_*\colon \pi_n(X,u(0))\to \pi_n(X,u(1))
    \]
    を定める.
    \item $u_*$は$u$の基点を止めたホモトピー類にしかよらない.
    \item $(v\cdot u)_*=v_*\circ u_*$\footnote{今回は,2つの道$u\colon x\to y,\ v\colon y\to z$に対し,それらの和の記号を$v\cdot u\colon x\to z$と書くことにする.}
    \item 以上によりtransport functor $\Pi(X)\to \mathbf{Grp}$を得る.
  \end{enumerate}
\end{prop}
\begin{proof}
  HEPをいっぱい使う.あとは略(多分かかない).
\end{proof}

相対バージョンも考える.$u\colon I\to {\color{red}A}$,$f\colon (I^n,\partial I^n,J^{n-1})\to (X,A,u(0))$が与えられたとする.やはり$u$はそれぞれ$u(0)$と$u(1)$に値を持つ2つの恒等写像$J^{n-1}\to X$の間のホモトピー$\hat u$を引き起こす.$(I^n,\partial I^n),\ (\partial I^n,J^{n-1})$はどちらもNDR-pairなので,$\hat u$は初期条件を$f$としてホモトピー$U\colon I^n \times I\to X$に延びる.

\begin{prop}(相対ホモトピー群に値を持つtransport functor)
  \begin{enumerate}[(1)]
    \item 対応$[f]=[U_0]\mapsto[U_1]$はwell-definedな準同型
    \[
      u_*\colon \pi_n(X,A,u(0))\to \pi_n(X,A,u(1))
    \]
    を定める.
    \item $u_*$は$u$の基点を止めたホモトピー類にしかよらない.
    \item $(v\cdot u)_*=v_*\circ u_*$.
    \item 以上によりtransport functor $\Pi(A)\to \mathbf{Grp}$を得る.
  \end{enumerate}
\end{prop}
\begin{proof}
  略(多分書かない).
\end{proof}

以上の結果から,$u_*(f)$とは,$f$を初期条件とする$u$に沿ったホモトピー$h$に対する,$[h(-,1)]$のことである.また$f$と$g$の間に,$u$に沿ったホモトピーがあれば$u_*f=g$である.

また,ホモトピー群への基本群の右作用
\[
  \beta\cdot\alpha:=\alpha_*(\beta)
\]が定まる.

Transport functorはもっと一般の状況でも定まる.\textbf{Set}に値を持つが.定義域がNDR-pair $(X,x_0)$なら同様にしてホモトピー集合に値を持つtransport functor $\Pi(Y)\to \mathbf{Set},\ y\to [(X,x_0),(Y,y_0)],\ \alpha\to \alpha_*$が定まる.

\paragraph{連続写像の引き起こす準同型との関係}
\begin{lem}
  写像$\alpha\colon (I,\partial I)\to (X,x_0)$, $f\colon (X,x_0)\to (Y,y_0)$に対し,次の図式
\[
  \xymatrix{
    \pi_n(X,x_0)\ar[r]^{f_*}\ar[d]_{\alpha_*}&\pi_n(Y,y_0)\ar[d]^{(f\circ \alpha)_*}\\
    \pi_n(X,x_0)\ar[r]_{f_*}&\pi_n(Y,y_0)
  }
\]
は可換である.
\end{lem}
\begin{proof}
  $[\varphi]\in\pi_(X,x_0)$に対し,$[\psi]=\alpha_*[\varphi]$とおく.このとき,$f_*\alpha_*[\varphi]=[f\circ \psi]$である.また,$\psi$のとりかたより,ホモトピー$h\colon I^{n}\times I\to X$ with $h(x,t)=\alpha(t)\ \ (x\in \partial I^n)$がとれる.一方,$(f\circ \alpha)_*\circ f_*[\varphi]=(f\circ \alpha)_*[f\circ \varphi]$は$[f\circ \psi]$である.実際,$f\circ h$は,$f\circ \varphi$と$f\circ \psi$の,$\alpha$に沿ったホモトピーを与える.
\end{proof}
相対バージョンもある.
\begin{lem}
  写像$\alpha\colon (I,\partial I)\to (A,x_0)$, $f\colon (A,x_0)\to (B,y_0)$に対し,次の図式
\[
  \xymatrix{
    \pi_n(A,x_0)\ar[r]^{f_*}\ar[d]_{\alpha_*}&\pi_n(Y,B,y_0)\ar[d]^{(f\circ \alpha)_*}\\
    \pi_n(A,x_0)\ar[r]_{f_*}&\pi_n(Y,B,y_0)
  }
\]
は可換である.
\end{lem}

\subsection{ホモトピー完全列との関係}
基本群の作用が対のホモトピー完全列に引き起こす準同型を見る.

\begin{thm}
  $(X,A,x_0)$を点付き空間対,$n\ge 0$とする.このとき各$[\gamma]=\alpha\in\pi_1(A)$に対し,次の図式は可換である.
  \[
    \xymatrix{
      \ar[r]
      &\pi_{n+1}(X,A)\ar[r]^{\partial_*}\ar[d]^{\alpha_*}
      &\pi_n(A)\ar[r]^{i_*}\ar[d]^{\alpha_*}
      &\pi_n(X)\ar[r]^{j_*}\ar[d]^{\alpha_*}
      &\pi_n(X,A)\ar[r]\ar[d]^{\alpha_*}
      &\\
      \ar[r]
      &\pi_{n+1}(X,A)\ar[r]^{\partial_*}
      &\pi_n(A)\ar[r]^{i_*}
      &\pi_n(X)\ar[r]^{j_*}
      &\pi_n(X,A)\ar[r]
      &
    }
  \]
\end{thm}
\begin{proof}
  連続写像の引き起こす準同型については可換性は上で確かめた.あとは左の四角についてにのみ確かめればよい.$f\colon (I^{n+1},\partial I^{n+1},J^n)\to (X,A,x_0)$を任意にとる.$\alpha_*[f]=[g]$なる$g\colon I^n\to X$をとる.このとき,$f$と$g$の$\gamma$に沿ったホモトピー$h\colon (I^{n+1},\partial I^{n+1},J^n)\times I\to (X,A,x_0)$がある.一方,$\alpha_*\partial_*[f]=\alpha_*[f|_{I^n\times 1}]$である.このとき,$h|_{I^n\times 1}\colon (I^n\times 1,\partial I^n\times 1)\to (A,x_0)$は,$f|_{I^n\times 1}$と$g|_{I^n\times 1}$の,$\gamma$に沿ったホモトピーを与える.よって,$\alpha_*\partial_*[f]=\alpha_*[f|_{I^n\times 1}]=[g|_{I^n\times 1}]=\partial_*[g]=\partial_*\alpha_*[f]$である.
\end{proof}
\section{Hurewiczの定理}
\section{Obstruction Cochainの定義}
\end{document}