\documentclass[a4paper,11pt]{jsarticle}

\usepackage{amsmath,amsthm,amsfonts,float,cases,bm,amssymb,amssymb,ascmac,url,color}
\usepackage[dvipdfmx]{graphicx}
\usepackage[all]{xy}
%\usepackage{makeidx}%これを使ったうえで\index{}を使わないとエラーになる
\usepackage[bbgreekl]{mathbbol}
\usepackage[%,
dvipdfmx,% 欧文ではコメントアウトする
setpagesize=false,%
bookmarks=true,%
bookmarksdepth=tocdepth,%
bookmarksnumbered=true,%
colorlinks=false,%
pdftitle={},%
pdfsubject={},%
pdfauthor={},%
pdfkeywords={}%
]{hyperref}% PDFのしおり機能の日本語文字化けを防ぐ((u)pLaTeXのときのみかく)
\usepackage{pxjahyper}
\DeclareMathSymbol{\bbepsilon}{\mathord}{bbold}{"0F}


%\makeindex%インデックスつかわないときにこれoffにしないとエラー出る

\setlength{\textwidth}{\fullwidth}
\setlength{\textheight}{40\baselineskip}
\addtolength{\textheight}{\topskip}
\setlength{\voffset}{-0.55in}

\theoremstyle{definition}
\newtheorem{thm}{定理}[section]
\newtheorem{prop}[thm]{命題}
\newtheorem{cor}[thm]{系}
\newtheorem{dfn}[thm]{定義}
\newtheorem{lem}[thm]{補題}
\newtheorem{rem}[thm]{注意}
\newtheorem{eg}[thm]{例}

\DeclareMathOperator{\Hom}{\mathrm{Hom}}
\DeclareMathOperator{\id}{\mathrm{id}}
\DeclareMathOperator{\Int}{\mathrm{Int}}

\setcounter{tocdepth}{3}%目次に表示する数字の深さ
\setcounter{section}{-1}

\begin{document}
\date{}
\title{Postnikov System}

\maketitle

%begin{abstract}
%end{abstract}

\tableofcontents

\section{ホモトピーファイバー列}
空間対やファイブレーションに対してホモトピー完全列が構成された.ここでは任意の連続写像に対してfibrant replacementを用いて,ファイブレーションに対するものと類似のホモトピー完全列を構成する.
\paragraph{ファイブレーションに対するホモトピー完全列}
$F\rightarrow E\rightarrow B$を基点付きファイブレーションとする.このとき,$k\ge 1$に対して$p_*\colon \pi_k(E,F)\to \pi_k(B)$が同型になるのだった.よって,対$(E,F)$のホモトピー完全列(下図上段)により,ファイブレーション$p$のホモトピー完全列(下図下段)を得る.
\[\xymatrix{
  \ar[r]&
  \pi_k(F)\ar@{=}[d]\ar[r] &
  \pi_k(E)\ar@{=}[d]\ar[r] &
  \pi_k(E,F)\ar[d]_{\cong}^{p_*}\ar[r]^{\partial_*} &
  \pi_{k-1}(F)\ar@{=}[d]\ar[r]&
  \\
  \ar[r]&
  \pi_k(F)\ar[r] &
  \pi_k(E)\ar[r]_{p_*} &
  \pi_k(B)\ar[r]_{\Delta_*} &
  \pi_{k-1}(F)\ar[r]&
}\]
ここで,$\Delta_*$は$\partial_*\circ (p_*)^{-1}$である.

\paragraph{fibrant replacement}
基点付き連続写像のホモトピーファイバー列を,ファイブレーションのホモトピー完全列を利用して作る.そのために,連続写像をファイブレーションに置き換える.
\begin{thm}(fibrant replacement)
  $f\colon (X,x_0)\to (Y,y_0)$を基点付き連続写像とする.また,\begin{align*}
    E_f&:=\{(x,\gamma)\in X\times Y^I\mid f(x)=\gamma(0)\}\\
    p_f&\colon E_f\to Y,\ (x,\gamma)\mapsto \gamma(1)\\
    r_f&\colon E_f\to X,\ (x,\gamma)\mapsto x\\
    F_f&:=p_f^{-1}(y_0)
  \end{align*}
  とおく.また$i_f$を包含$F_f\to E_f$,$\pi_f$を合成$r\circ i_f$とおく.
  
  このとき$p_f\colon (E_f,(x_0,c_{y_0}))\to(Y,y_0)$は基点付きファイブレーションで下の図式(特に下の三角形)はホモトピー可換である.
  \[\xymatrix{
    &&F_f\ar@{_(->}[d]^{i_f}\ar[lld]_{\pi_f}\\
    X\ar[dr]_f & \ar@{}[ur]|{\circlearrowright}&E_f\ar[ll]^{r_f} \ar[dl]^{p_f}\\
    &Y&
  }\]
\end{thm}
  \begin{proof}
    略.
  \end{proof}
  \begin{thm}$f\colon X\to Y$を基点付き連続写像とする.$Y$のpath fibrationを$f$で引き戻したものを$\pi'_f\colon F'_f\to X$とおく(左下図式).このとき,右下の図式が可換になるような同相写像$\varphi\colon F_f\to F'_f$がある.
    \[
    \xymatrix{
      F'_f\ar[d]_{\pi'_f}\ar[r]&P(Y,y_0)\ar[d]^{\mathrm{ev}_1}&F_f\ar[r]^{\varphi}_{\cong}\ar[d]_{\pi_f}&F'_f\ar[d]^{\pi'_f}\\
      X\ar[r]_f&Y&X\ar@{=}[r]&X
    }  
    \]
    特に,$\pi_f$はファイブレーションである.
  \end{thm}
  \begin{proof}
    略.$F_f$と$F'_f$を具体的に書き下せば$\varphi$をどう作ればよいかわかる.
  \end{proof}
\paragraph{連続写像のホモトピー完全列}
$f\colon (X,x_0)\to (Y,y_0)$を基点付き連続写像とし,$F:=\pi_f^{-1}(x_0)$($\pi_f$は前段落のもの)とおき,$F$の$F_f$への包含を$j_f$とおく.$f$のfibrant replacement $p_f$のホモトピー完全列と$\pi_f$のホモトピー完全列を組み合わせて下の図式を得る.
\[\xymatrix{
  \ar[r]&
  \pi_{k+1}(Y)\ar[r]^{\Delta_*}&
  \pi_{k}(F_f)\ar[r]^{(i_f)_*}\ar@{=}[d]&
  \pi_k(E_f)\ar[r]^{(p_f)_*}\ar[d]^{r_*}_{\cong}&
  \pi_{k}(Y)\ar[r]^{\Delta_*}&
  \pi_{k-1}(F_f)\ar[r]\ar@{=}[d]&
  \\
  \ar[r]&
  \pi_k(F)\ar[r]_{(j_{f})_*}&
  \pi_k(F_f)\ar[r]_{(\pi_f)_*}&
  \pi_k(X)\ar[r]_{\Delta_*}\ar[ru]_{f_*}&
  \pi_{k-1}(F)\ar[r]_{(j_{f})_*}&
  \pi_{k-1}(F_f)\ar[r]&
}\]
上の図式の台形の部分以外は,連続写像のレベルでのホモトピー可換図式から引き起こされるので可換である.よって,完全列
\[\xymatrix{
  \ar[r]&
  \pi_{k+1}(Y)\ar[r]^{\Delta_*}&
  \pi_k(F_f)\ar[r]^{(\pi_f)_*}&
  \pi_k(X)\ar[r]^{f_*}&
  \pi_k(Y)\ar[r]^{\Delta_*}&
  \pi_{k-1}(F_f)\ar[r]&
}\]
を得る.これを$f$のホモトピファイバー列という.
\bibliographystyle{alpha}
\bibliography{bib}

%printindex

\end{document}