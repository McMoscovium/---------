\documentclass[a4paper,11pt]{jsarticle}

\usepackage{amsmath,amsthm,amsfonts,float,cases,bm,amssymb,amssymb,ascmac,url,color,enumerate}
\usepackage[dvipdfmx]{graphicx}
\usepackage[all]{xy}
%\usepackage{makeidx}%これを使ったうえで\index{}を使わないとエラーになる
\usepackage[bbgreekl]{mathbbol}
\usepackage[%,
dvipdfmx,% 欧文ではコメントアウトする
setpagesize=false,%
bookmarks=true,%
bookmarksdepth=tocdepth,%
bookmarksnumbered=true,%
colorlinks=false,%
pdftitle={},%
pdfsubject={},%
pdfauthor={},%
pdfkeywords={}%
]{hyperref}% PDFのしおり機能の日本語文字化けを防ぐ((u)pLaTeXのときのみかく)
\usepackage{pxjahyper}
\DeclareMathSymbol{\bbepsilon}{\mathord}{bbold}{"0F}


%\makeindex%インデックスつかわないときにこれoffにしないとエラー出る

\setlength{\textwidth}{\fullwidth}
\setlength{\textheight}{40\baselineskip}
\addtolength{\textheight}{\topskip}
\setlength{\voffset}{-0.55in}

\theoremstyle{definition}
\newtheorem{thm}{定理}[section]
\newtheorem{prop}[thm]{命題}
\newtheorem{cor}[thm]{系}
\newtheorem{dfn}[thm]{定義}
\newtheorem{lem}[thm]{補題}
\newtheorem{rem}[thm]{注意}
\newtheorem{eg}[thm]{例}

\DeclareMathOperator{\Hom}{\mathrm{Hom}}
\DeclareMathOperator{\id}{\mathrm{id}}
\DeclareMathOperator{\Int}{\mathrm{Int}}

\setcounter{tocdepth}{3}%目次に表示する数字の深さ
\setcounter{section}{-1}

\begin{document}
\date{}
\title{Postnikov System}

\maketitle

\begin{abstract}
  Griffiths-Morganの教科書のPostnikov Systemの構成がちゃんと条件を満たすことを示せなかった.主に$f_n$の引き起こすホモトピー群の準同型の記述に困難性があった.前回終盤の議論により,$K(\pi,n)$をcodomainにもつホモトピー集合$[X,K(\pi,n+1)]$と,$X$上のprincipal $K(\pi,n)$ファイブレーション(with a section)の同値類の対応,および同一視$K(\pi_n(X),n)=\pi_n(X)$の扱いに気を付けながら丁寧に構成を追う必要があると思われた.

  これを解決するために,別の文献(G.W.Whitehead)でのPostnikov Systemの構成を見て,Eilenberg-MacLane空間の扱いを学ぶとともに,Griffiths-Morganで直面した$f_{n*}$を記述する手法を模索する.
\end{abstract}

\tableofcontents

\section{ホモトピーファイバー列}
空間対やファイブレーションに対してホモトピー完全列が構成された(河澄等を参照).ここでは記号の準備もかねて任意の連続写像に対してfibrant replacementを用いて,ファイブレーションに対するものと類似のホモトピー完全列を構成する.
\paragraph{ファイブレーションに対するホモトピー完全列}
$F\rightarrow E\rightarrow B$を基点付きファイブレーションとする.このとき,$k\ge 1$に対して$p_*\colon \pi_k(E,F)\to \pi_k(B)$が同型になるのだった.よって,対$(E,F)$のホモトピー完全列(下図上段)により,ファイブレーション$p$のホモトピー完全列(下図下段)を得る.
\[\xymatrix{
  \ar[r]&
  \pi_k(F)\ar@{=}[d]\ar[r] &
  \pi_k(E)\ar@{=}[d]\ar[r] &
  \pi_k(E,F)\ar[d]_{\cong}^{p_*}\ar[r]^{\partial_*} &
  \pi_{k-1}(F)\ar@{=}[d]\ar[r]&
  \\
  \ar[r]&
  \pi_k(F)\ar[r] &
  \pi_k(E)\ar[r]_{p_*} &
  \pi_k(B)\ar[r]_{\Delta_*} &
  \pi_{k-1}(F)\ar[r]&
}\]
ここで,$\Delta_*$は$\partial_*\circ (p_*)^{-1}$である.

\paragraph{fibrant replacement}
基点付き連続写像のホモトピーファイバー列を,ファイブレーションのホモトピー完全列を利用して作る.そのために,連続写像をファイブレーションに置き換える.
\begin{thm}[fibrant replacement]
  $f\colon (X,x_0)\to (Y,y_0)$を基点付き連続写像とする.また,\begin{align*}
    E_f&:=\{(x,\gamma)\in X\times Y^I\mid f(x)=\gamma(0)\}\\
    p_f&\colon E_f\to Y,\ (x,\gamma)\mapsto \gamma(1)\\
    r_f&\colon E_f\to X,\ (x,\gamma)\mapsto x\\
    F_f&:=p_f^{-1}(y_0)
  \end{align*}
  とおく.また$i_f$を包含$F_f\to E_f$,$\pi_f$を合成$r\circ i_f$とおく.
  
  このとき$p_f\colon (E_f,(x_0,c_{y_0}))\to(Y,y_0)$は基点付きファイブレーションで下の図式(特に下の三角形)はホモトピー可換である.
  \[\xymatrix{
    &&F_f\ar@{_(->}[d]^{i_f}\ar[lld]_{\pi_f}\\
    X\ar[dr]_f & \ar@{}[ur]|{\circlearrowright}&E_f\ar[ll]^{r_f} \ar[dl]^{p_f}\\
    &Y&
  }\]
\end{thm}
  \begin{proof}
    略.玉木大のファイバー束とホモトピーを参照.
  \end{proof}
  \begin{lem}$f\colon X\to Y$を基点付き連続写像とする.$Y$のpath fibrationを$f$で引き戻したものを$\pi'_f\colon F'_f\to X$とおく(左下図式).このとき,右下の図式が可換になるような同相写像$\varphi\colon F_f\to F'_f$がある.
    \[
    \xymatrix{
      F'_f\ar[d]_{\pi'_f}\ar[r]&P(Y,y_0)\ar[d]^{\mathrm{ev}_1}&F_f\ar[r]^{\varphi}_{\cong}\ar[d]_{\pi_f}&F'_f\ar[d]^{\pi'_f}\\
      X\ar[r]_f&Y&X\ar@{=}[r]&X
    }  
    \]
    特に,$\pi_f$はファイブレーションであり,$\pi_f$と$\pi'_f$はX上で同型である.
  \end{lem}
  \begin{proof}
    略.$F_f$と$F'_f$を具体的に書き下せば$\varphi$をどう作ればよいかわかる.
  \end{proof}
\paragraph{連続写像のホモトピー完全列}
\begin{thm}[ホモトピーファイバー列]
  $f\colon (X,x_0)\to (Y,y_0)$を基点付き連続写像とし,$F:=\pi_f^{-1}(x_0)$($\pi_f$は前段落のもの)とおき,$F$の$F_f$への包含を$j_f$とおく.$f$のfibrant replacement $p_f$のホモトピー完全列と$\pi_f$のホモトピー完全列を組み合わせて下の図式を得る.
\[\xymatrix{
  \ar[r]&
  \pi_{k+1}(Y)\ar[r]^{\Delta_*}&
  \pi_{k}(F_f)\ar[r]^{(i_f)_*}\ar@{=}[d]&
  \pi_k(E_f)\ar[r]^{(p_f)_*}\ar[d]^{r_*}_{\cong}&
  \pi_{k}(Y)\ar[r]^{\Delta_*}&
  \pi_{k-1}(F_f)\ar[r]\ar@{=}[d]&
  \\
  \ar[r]&
  \pi_k(F)\ar[r]_{(j_{f})_*}&
  \pi_k(F_f)\ar[r]_{(\pi_f)_*}&
  \pi_k(X)\ar[r]_{\Delta_*}\ar[ru]_{f_*}&
  \pi_{k-1}(F)\ar[r]_{(j_{f})_*}&
  \pi_{k-1}(F_f)\ar[r]&
}\]
上の図式の台形の部分以外は,連続写像のレベルでのホモトピー可換図式から引き起こされるので可換である.よって,完全列
\[\xymatrix{
  \ar[r]&
  \pi_{k+1}(Y)\ar[r]^{\Delta_*}&
  \pi_k(F_f)\ar[r]^{(\pi_f)_*}&
  \pi_k(X)\ar[r]^{f_*}&
  \pi_k(Y)\ar[r]^{\Delta_*}&
  \pi_{k-1}(F_f)\ar[r]&
}\]
を得る.これを$f$のホモトピファイバー列という.
\end{thm}


\section{Postnikov Systemの構成}

\subsection{G. W. Whiteheadの教科書の構成}
Griffiths-Morganの教科書にある構成で本当にPostnikov Syaytemの条件を満たすものが作れているかチェックしきれなかった.とくに$f_n$の引き起こすホモトピー群の同型の部分が示せなかった.

そこでまず,G. W. Whiteheadの教科書にあるPostnikov systemの構成を詳しく調べる.これはGriffiths-Morganにあるものとは手順が少しだけ違う.まず,与えられた空間$X$に対し,$K(\pi_n(X),n)$-fibrationのタワーを先に作り,そのあとfibrationのsectionに対する障害理論を使わずに各$f_n\colon X\to X_n$を構成するという手を取っている.この構成とGriffiths-Morganの教科書の構成に類似点を見つけ,Griffiths-Morganの教科書の構成の方を理解しようとしてみる.
\paragraph{$n$-connective fibration}
$X$を弧状連結空間とする.このとき,$\pi_{n+1}(X)$の生成元に沿って$X$に$(n+2)$-cellを接着することで新たな空間$X'$で$\pi_{n+1}(X')=0$なるものが作れる.これを繰り返すことで$(n+1)$-連結な相対CW複体$(X^*,X)$であって$\pi_i(X^*)=0$ for $i\ge n+1$を満たすものが作れる.
\[
\xymatrix{
  \ar[r]&
  \pi_{k+1}(X^*)\ar[r]&
  \pi_{k+1}(X^*,X)\ar[r]&
  \pi_{k}(X)\ar[r]&
  \pi_k(X^*)\ar[r]&
  \pi_k(X^*,X)\ar[r]&
  \pi_k(X)\ar[r]&
}  
\]
さらに,包含$i\colon X\hookrightarrow X^*$に対し$X_n:=F_i$,$p_n:=\pi_i$とおく.記号はfibrant replacementの定義で使われているものに準ずる.$i$のファイバーホモトピー列より,ファイブレーション$p_n$は次を満たす.
\[\xymatrix{
  \ar[r]&
  \pi_{k+1}(X^*)\ar[r]^{\Delta_*}&
  \pi_k(X_n)\ar[r]^{p_{n*}}&
  \pi_k(X)\ar[r]&
  \pi_k(X^*)\ar[r]&
}\]
\begin{itemize}
  \item $\pi_k(X^*)=\pi_{k+1}(X^*)=0$ for $k\ge n+1$ゆえ$p_{n*}$は$k\ge n+1$で同型.
  \item $X^*$は$n+1$次元以下のセルを持たないので,包含準同型$\pi_{k}(X)\to \pi_k(X^*)$は$k\le n$で同型,$k=n+1$で全射.よって$X_n$は$n$-連結である.
\end{itemize}
以上をまとめると,次を得る.
\begin{thm}[$n$-connective fibrationの存在]\label{existenceOfConnFib}
  $X$を弧状連結空間とする.このとき,任意の$n>0$に対し,ファイブレーション$p_n\colon X_n\to X$で次を満たすものが存在する.
  \begin{itemize}
    \item $p_{n*}\colon \pi_k(X_n)\to \pi_k(X)$は$k\ge n+1$で同型.
    \item $X_n$は$n$-連結.
  \end{itemize}
\end{thm}
定理の条件を満たすファイブレーションを$n$-connective fibrationと呼ぶ.また,$\pi_i(Y)=0\ (i\le n)$を満たす空間$Y$を$n$-anticonnected spaceと呼ぶ.$X$を含む空間$X^*$が$n$-anticonnectedで$(X^*,X)$が$n$-connectedなら$X^*$を$X$の$n$-anticonnected extensionと呼ぶ.さらに加えて$(X^*,X)$が相対CW-複体で,$X^*$が$n$次元以下のセルを持たないとき,$X^*$は$X$のregular $n$-anticonnected extensionと呼ぶ.

とくに,上の定理の$X_n$を構成するときに使った$X^*$は,$X$のregular $(n+1)$-anticonnected extensionである.

以下いくつかの命題を用いて,\ref{existenceOfConnFib}で構成された$n$-connective fibrationのホモトピー一意性を示す.

\begin{thm}[連続写像をそのanticonnected extensionに延ばすための十分条件と延長のホモトピー一意性]
  $X^*\ (\mathrm{resp.\ }Y^*)$をそれぞれ$X\ (\mathrm{resp.\ }Y)$のregular $m\ (\mathrm{resp.\ }n)$-anticonnected extensionとし,$f\colon X\to Y$を連続写像とする.このとき次が成立する.
  \begin{enumerate}[(1)]
    \item $m\le n$ならば,$f$は$\tilde{f}\colon X^*\to Y^*$に延びる.
    \item $m\le n+1$ならば,2つの延長$g,g'\colon X^*\to Y^*$はホモトピック(rel. $X$)である.
  \end{enumerate}  
\end{thm}
\begin{proof}
  \begin{enumerate}[(1)]
    \item $X^*$は$n$次元以下のセルを持たないので,$f$は$n$-skeleton $(X^*)^{(n)}=X$に延びる.$f$を$(X^*)^{(k)}\to Y^*$に延長する障害類は$H^{k+1}(X^*,X;\pi_k(Y^*))=0$に属する($k\ge n$).よって$f$は$X^*$に延長する.
    \item $g,g'$を$f$の延長とする.これらは$(X^*)^{(n)}$上で$f$に一致する.$k\le n+1$とする.$g,g'$のホモトピー(rel. $X$)を$X^{(k)}$に延ばすための障害類は$H^k(X^*,X;\pi_k(Y^*))=0$に属するので,$g,g'$はホモトピック(rel. $X$)である.
  \end{enumerate}
\end{proof}
\begin{cor}[anticonnected extensionの一意性]
$X^*,X'^*$をともに$X$のregular $n$-anticonnected extensionとする.このとき,$(X^*,X)$と$(X'^*,X)$はホモトピー同値である.
\end{cor}
\begin{proof}
  定理で$m=n$としたものより,$X$上の恒等写像は$f\colon (X^*,X)\to (X'^*,X),\ g\colon (X'^*,X)\to (X^*,X)$に延びる.合成$f\circ g,\ g\circ f$はともに$X$上の恒等写像の延長だから,定理よりこれらは恒等写像にホモトピック(rel. $X$)である.よって対のホモトピー同値$(X^*,X)\simeq (X'^*,X)$を得る.
\end{proof}
\begin{cor}[\ref{existenceOfConnFib}で構成された$n$-connective fibrationの一意性]

  $X^*,X'^*$を,$X$のregular $(n+1)$-connected extensionとする.$p\colon \Tilde{X}\to X,\ p'\colon \tilde{X'}\to X$をそれぞれ定理\ref{existenceOfConnFib}で構成した$n$-connective fibrationとする\footnote{つまり,$p$は$X^*$のpath fibrationのinclusionによる引き戻しである.}.このとき,$p,p'$はファイバーホモトピー同値である.
\end{cor}
\begin{proof}
  $f\colon (X^*,X)\to (X'^*,X),\ g\colon (X'^*,X)\to (X^*,X)$をそれぞれ$\id_X$の延長とする.$f_*$を$f$が誘導するpointed path space\footnote{$X^*$と$X'^*$の基点は共通の部分集合$X$の点を取ることにする.}上の射$PX^*\to PX'^*$とするとpullbackの普遍性より$f_1\colon \tilde{X}\to\tilde{X'}$が伸びる.
  \[\xymatrix{
    \tilde{X}\ar@{.>}[dr]^{f_1}\ar[ddr]_{p}\ar[rr]^{\mathrm{pr}_2}&&PX^*\ar[d]^{f_*}\\
    & \tilde{X'}\ar[d]_{p'}\ar[r]&PX'^*\ar[d]^{\mathrm{ev_1}}\\
    &X\ar@{^(->}[r]&X'^*
  }\]同様にして$g_1\colon \tilde{X'}\to\tilde{X}$を得る.$H\colon X^*\times I\to X^*$を$\id_{X^*}$と$g\circ f$のホモトピー(rel. $X$)とすると,$H_1((x,\gamma),t):=(x,H(\gamma(-),t))$によりホモトピー$H_1\colon\tilde{X}\times I\to \tilde X$が定まる.これは$\id_{\tilde{X}}$と$g_1\circ f_1$の,$X$上のホモトピーである.同様に,$\id_{\tilde{X'}}$と$f_1\circ g_1$の$X$上のホモトピーもあるので,$p$と$p'$はファイバーホモトピー同値である.
\end{proof}
\bibliographystyle{alpha}
\bibliography{bib}

%printindex

\end{document}