\documentclass[a4paper,11pt]{jsarticle}


\usepackage{amsmath,amsthm,amsfonts,float,cases,bm,amssymb,amssymb,ascmac}
\usepackage[dvipdfmx]{graphicx}
\usepackage[all]{xy}
\usepackage{makeidx}
\makeindex

\setlength{\textwidth}{\fullwidth}
\setlength{\textheight}{40\baselineskip}
\addtolength{\textheight}{\topskip}
\setlength{\voffset}{-0.55in}

\newtheorem{thm}{定理}[section]
\newtheorem{cor}[thm]{系}
\newtheorem{dfn}[thm]{定義}
\newtheorem{lem}[thm]{補題}
\newtheorem{rem}[thm]{注意}
\newtheorem{eg}[thm]{例}


\begin{document}

\title{ホモトピー論の基本について}
\maketitle
\begin{abstract}
  ホモトピー論の基本について網羅的に書いたものではなく,勉強中に特に注意して考えたことをまとめたもの.
\end{abstract}
\tableofcontents
\section{ホモトピー群}
\subsection{相対ホモトピー群の2種類の定義}
$I^n$の部分空間$J^n$を\index{$J^n$}$J^n:=\partial I^{n-1}\times I\cup I^{n-1}\times\{0\}$と定める.$n=1,\ 2,\ 3$の場合を書けば形がわかる.$n=1$の場合$\{0\}$,$n=2$の場合上が空いたコの字型($\sqcup$),$n=3$の場合上の開いた箱型である.

\index{きてんつきくうかんつい@基点付き空間対}\textgt{基点付き空間対}$(X,A,*)$とは空間対$(X,A)$と$A$の点$*\in A$の組のことである.このとき,$n>0$に対し$(X,A,*)$の\index{ほもとぴーぐん@($n$次)ホモトピー群}\textgt{$n$次ホモトピー群}$\pi_n(X,A,*)$を3対のホモトピー集合\[
  \pi_n(X,A,*):=[(I^n,\partial I^n,J^n),(X,A,*)]
\]と定める.
\subsection{Hurewiczの定理}
\subsection{Freudenthalの懸垂定理}
\subsection{Whiteheadの定理}
\section{局所係数の(コ)ホモロジー}
\subsection{局所系の2種類の定義}
\subsection{局所係数の(コ)ホモロジーの計算例}
\section{障害理論}
\subsection{CW複体上の写像を延ばせるかどうかのcriterion}
\subsection{E-M空間}

\printindex





\end{document}