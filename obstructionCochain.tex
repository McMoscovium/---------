\documentclass[a4paper,11pt]{jsarticle}


\usepackage{amsmath,amsthm,amsfonts,float,cases,bm,amssymb,amssymb,ascmac,url,color}
\usepackage[dvipdfmx]{graphicx}
\usepackage[all]{xy}
\usepackage{makeidx}
\usepackage[bbgreekl]{mathbbol}
\usepackage[%
 dvipdfmx,% 欧文ではコメントアウトする
 setpagesize=false,%
 bookmarks=true,%
 bookmarksdepth=tocdepth,%
 bookmarksnumbered=true,%
 colorlinks=false,%
 pdftitle={},%
 pdfsubject={},%
 pdfauthor={},%
 pdfkeywords={}%
]{hyperref}
% PDFのしおり機能の日本語文字化けを防ぐ((u)pLaTeXのときのみかく)
\usepackage{pxjahyper}
\DeclareMathSymbol{\bbepsilon}{\mathord}{bbold}{"0F}



\setlength{\textwidth}{\fullwidth}
\setlength{\textheight}{40\baselineskip}
\addtolength{\textheight}{\topskip}
\setlength{\voffset}{-0.55in}

\theoremstyle{definition}
\newtheorem{thm}{定理}[section]
\newtheorem{prop}[thm]{命題}
\newtheorem{cor}[thm]{系}
\newtheorem{dfn}[thm]{定義}
\newtheorem{lem}[thm]{補題}
\newtheorem{rem}[thm]{注意}
\newtheorem{eg}[thm]{例}

\DeclareMathOperator{\Hom}{\mathrm{Hom}}
\DeclareMathOperator{\id}{\mathrm{id}}
\DeclareMathOperator{\Int}{\mathrm{Int}}

\setcounter{tocdepth}{3}%目次に表示する数字の深さ
\setcounter{section}{-1}

\begin{document}
\textgt{編集メモ}
\begin{itemize}
  \item このメモは昼間眠くないときにこのpdfを読んで気づいたことをまとめたものである.夜に編集すると勢いでやっつけ仕事のようになる場合がある.
  \item 現状分かりにくい.\begin{itemize}
    \item 「便利な写像」は便利だが,基本群のホモトピー群への作用をここまで具体的に書かなくても後の議論は回るのでは?
    \item 絶対ホモトピー群は$[(S^n,*),(X,*)]_0$を採用していながら相対ホモトピー群は$[(I^n,\partial I^n,J^{n-1}),(X,A,*)]_0$を採用している部分がある.後者のcubeを使う方法にした方が対のホモトピー完全列を記述しやすい.
  \end{itemize}
\end{itemize}

\date{}
\title{Obstruction Cochain}

\maketitle

\begin{abstract}
  Obstruction cochainが特性写像のとりかたに依らないこと,およびそれがcocycleであることの証明
\end{abstract}
\tableofcontents
\section{基本群のホモトピー群への作用}
\subsection{便利な写像}
$C_i$でinclusion $i\colon S^{n-1}\to D^n$のmapping cylinderを表すことにする.ただし,$CS^{n-1}=I\times S^{n-1}/~$の$1\times S^{n-1}$と$\partial D^n$を接着しているとする.同相写像$C_i\to D^n$を次のように作る.\[
  \begin{alignedat}{2}
  (t,x)&\mapsto \frac{2-t}{2}x&\ \ \ &((t,x)\in CS^{n-1})\\
  x&\mapsto x/2&&(x\in D^n)
\end{alignedat}
\]これの逆を$b\colon D^n\to C_i$とおく.
\subsection{基本群のホモトピー群への作用}
\paragraph{絶対ホモトピー群への作用}
G. W. WhiteheadのIII章を参照\footnote{一度立ち止まって,この文章を声に出して読め.}(この小節の内容よりも一般の状況でいろいろ書いてある).

$X$の点$x_1,x_0$を結ぶpath $\gamma\colon (I,0,1)\to (X,x_1,x_0)$の,ホモトピー群$\pi_n(X,x_0)$への作用を定める.spheroid\footnote{球面からの連続写像をspheroidと呼ぶことにする.}$f\colon (D^n,S^{n-1})\to (X,x_0)$に対し,$\tau_\gamma (f)\colon (D^n,S^{n-1})\to (X,x_1)$を$\tau_\gamma(f):=((\id_I\times \gamma)\cup f)\circ b$と定める\footnote{$f\colon CS^{n-1}\to X$と$g\colon D^n\to X$で,$f|_{1\times S^{n-1}}=g|_{\partial D^n}$を満たすものが引き起こす写像$C_i\to X$のことを$f\cup g$と書くことにする.}.
\begin{lem}
  $\gamma,\gamma'$が端点を止めてホモトピック,$f,f'$が基点を止めてホモトピックであるとする.このとき,$\tau_\gamma(f)$と$\tau_{\gamma'}(f')$は基点を止めてホモトピックである.
\end{lem}
\begin{proof}
  $\gamma,\gamma'$のホモトピーを$h\colon I\times I\to X$,$f,f'$のホモトピーを$H\colon I\times D^n\to X$とおく.このとき,ホモトピー$\Phi\colon I\times D^n\to X$を,$\Phi_t:=((\id_I\times h_t)\cup H_t)\circ b$と定める.これを具体的に書くと
  \[\begin{cases}
    x\mapsto H(t,2x)&(0\le \|x\|\le 1/2)\\
    x\mapsto (2-2\|x\|,h(t,x/\|x\|))&(1/2\le\|x\|\le 1)
  \end{cases}
  \]である.これは連続で,境界は基点にうつる.また$\Phi_0=\tau_\gamma(f)$,$\Phi_1=\tau_{\gamma'}(f')$である.
\end{proof}
この補題から,$\gamma$の端点を止めたホモトピー類$\alpha=[\gamma]$は写像$\tau_\alpha \colon\pi_n(X,x_0)\to \pi_n(X,x_1)$を定める.
\paragraph{相対ホモトピー群への作用}
$f\colon(D^n,S^{n-1},*)\to (X,A,x_0)$とpath $\gamma\colon (I,0,1)\to (A,x_1,x_0)$を任意にとる.$\tau'_\gamma(f)\colon (D^n,S^{n-1},*)\to (X,A,x_1)$を$\tau'_\gamma(f):=(\gamma\vee f)\circ b'$と定める.
\begin{lem}
  $\gamma,\gamma'$が端点を止めてホモトピック,$f,f'$が$\pi_n(X,A,x_0)$の同じ元を代表するとする.このとき,$\tau'_\gamma(f)$と$\tau'_{\gamma'}(f')$は$\pi_n(X,A,x_1)$の中で同じである.
\end{lem}
\begin{proof}
  絶対バージョンの同じ補題とパラレルである.
\end{proof}
上の補題から,$\gamma$の端点を止めた$A$の中でのホモトピー類$\alpha$は写像$\tau'_\alpha\colon\pi_n(X,A,x_0)\to \pi_n(X,A,x_1)$を定める.
\subsection{ホモトピー完全列との関係}
基本群の作用が対のホモトピー完全列に引き起こす準同型を見る.inclusionの引き起こす準同型$j_*$を$j_*\colon\pi_n(X)\to \pi_n(X,A)$を,$j_*[f]:=[f\circ p]$と定める.ここで,$p$は$p\colon (D^n,S^{n-1})\to(D^n/S^{n-1},S^{n-1}/S^{n-1})=(S^n,*)$である.あとの$i_*$や$\partial_*$は自明なとおりである.

\begin{thm}
  $(X,A)$を空間対,$n\ge 0$とする.このとき各$[\gamma]=\alpha\in\pi_1(A)$に対し,次の図式は可換である.ただし,$\beta:=i_*\alpha$である.
  \[
    \xymatrix{
      \ar[r]
      &\pi_{n+1}(X,A)\ar[r]^{\partial_*}\ar[d]^{\tau'_\alpha}
      &\pi_n(A)\ar[r]^{i_*}\ar[d]^{\tau_\alpha}
      &\pi_n(X)\ar[r]^{j_*}\ar[d]^{\tau_\beta}
      &\pi_n(X,A)\ar[r]\ar[d]^{\tau'_\alpha}
      &\\
      \ar[r]
      &\pi_{n+1}(X,A)\ar[r]^{\partial_*}
      &\pi_n(A)\ar[r]^{i_*}
      &\pi_n(X)\ar[r]^{j_*}
      &\pi_n(X,A)\ar[r]
      &
    }
  \]
\end{thm}
\begin{proof}
  (左の四角について)$[f]\in\pi_{n+1}(X,A)$を任意にとる.右上からたどる合成について,
  \[
    \tau_\alpha\partial_*[f]=\tau_\gamma(|_{S^n})=(\gamma\vee f|_{S^n})\circ b
  \]
  である.また,左下の合成は
  \[
    \partial_*\tau'_\alpha[f]=\partial_*((\gamma\vee f)\circ b')=(\gamma\vee f)\circ b=(\gamma\vee f|_{S^n})\circ b
  \]だから左の四角は可換である.

  (真ん中の四角について)$[f]\in \pi_n(A)$を任意にとる.右上からたどる合成について,
  \[
    \tau_\beta i_*[f]=\tau_{i_*\alpha}[f]=((i\circ\gamma)\vee f)\circ b=(\gamma\vee f)\circ b
  \]
  左下をたどる合成について,
  \[
    i_*\tau_\alpha[f]=i_*((\gamma\vee f)\circ b)=(\gamma\vee f)\circ b
  \]よって真ん中は可換である.

  (右の四角について)$[f]\in\pi_n(X)$を任意にとる.右上をたどる合成について,
  \[
    \tau'_\alpha j_*[f]=\tau'_\gamma(f\circ p)=(\gamma\vee (f\circ p))\circ b'
  \]左下をたどる合成について,
  \[
    j_*\tau_\beta[f]=j_*\tau_\gamma(f)=j_*((\gamma\vee f)\circ b)=(\gamma\vee f)\circ b\circ p
  \]
\end{proof}
\begin{lem}$\gamma*$は群準同型である.
\end{lem}
\begin{proof}
  $[f],[g]\in \pi_n(X,x_1)$をとる.
\end{proof}
\end{document}